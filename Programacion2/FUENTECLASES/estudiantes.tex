# Función para guardar los datos en un archivo de texto
def guardar_en_archivo(diccionario, nombre_archivo):
    with open(nombre_archivo, "w", encoding="utf-8") as archivo:
        for nombre, datos in diccionario.items():
            archivo.write(f"Nombre: {nombre}\n")
            archivo.write(f"Edad: {datos['edad']}\n")
            archivo.write("Materias:\n")
            for materia in datos["materias"]:
                archivo.write(f" - {materia}\n")
            archivo.write("\n")  # línea en blanco entre estudiantes
    print(f"Datos guardados en '{nombre_archivo}'")

    guardar_en_archivo(diccionario_estudiantes, "estudiantes.txt")

    # Tupla inicial
tupla_estudiantes = (
    ("Ana", 20, ["Matemática", "Programación", "Inglés"]),
    ("Luis", 22, ["Estadística", "Bases de Datos", "Inglés Técnico"]),
    ("María", 19, ["Lógica", "Probabilidad", "Programación"])
)

# Convertir la tupla en lista
lista_estudiantes = list(tupla_estudiantes)

# Agregar nuevo estudiante
nuevo_estudiante = ("Carlos", 21, ["Álgebra", "Física", "Inglés"])
lista_estudiantes.append(nuevo_estudiante)

# Transformar en diccionario
diccionario_estudiantes = {}
for nombre, edad, materias in lista_estudiantes:
    diccionario_estudiantes[nombre] = {
        "edad": edad,
        "materias": materias
    }

# Mostrar los datos
print("=== DATOS DE LOS ESTUDIANTES ===")
for nombre, datos in diccionario_estudiantes.items():
    print(f"\nNombre: {nombre}")
    print(f"Edad: {datos['edad']}")
    print("Materias:")
    for materia in datos["materias"]:
        print(f" - {materia}")

# Función para guardar los datos en un archivo de texto
def guardar_en_archivo(diccionario, nombre_archivo):
    with open(nombre_archivo, "w", encoding="utf-8") as archivo:
        for nombre, datos in diccionario.items():
            archivo.write(f"Nombre: {nombre}\n")
            archivo.write(f"Edad: {datos['edad']}\n")
            archivo.write("Materias:\n")
            for materia in datos["materias"]:
                archivo.write(f" - {materia}\n")
            archivo.write("\n")
    print(f"\nDatos guardados en '{nombre_archivo}'")

# Guardar en archivo
guardar_en_archivo(diccionario_estudiantes, "estudiantes.txt")

# Tupla inicial
tupla_estudiantes = (
    ("Ana", 20, ["Matemática", "Programación", "Inglés"]),
    ("Luis", 22, ["Estadística", "Bases de Datos", "Inglés Técnico"]),
    ("María", 19, ["Lógica", "Probabilidad", "Programación"])
)

# Convertir la tupla en lista
lista_estudiantes = list(tupla_estudiantes)

# Agregar nuevo estudiante
nuevo_estudiante = ("Carlos", 21, ["Álgebra", "Física", "Inglés"])
lista_estudiantes.append(nuevo_estudiante)

# Transformar en diccionario
diccionario_estudiantes = {}
for nombre, edad, materias in lista_estudiantes:
    diccionario_estudiantes[nombre] = {
        "edad": edad,
        "materias": materias
    }

# Mostrar los datos
print("=== DATOS DE LOS ESTUDIANTES ===")
for nombre, datos in diccionario_estudiantes.items():
    print(f"\nNombre: {nombre}")
    print(f"Edad: {datos['edad']}")
    print("Materias:")
    for materia in datos["materias"]:
        print(f" - {materia}")

# Función para guardar los datos en un archivo de texto
def guardar_en_archivo(diccionario, nombre_archivo):
    with open(nombre_archivo, "w", encoding="utf-8") as archivo:
        for nombre, datos in diccionario.items():
            archivo.write(f"Nombre: {nombre}\n")
            archivo.write(f"Edad: {datos['edad']}\n")
            archivo.write("Materias:\n")
            for materia in datos["materias"]:
                archivo.write(f" - {materia}\n")
            archivo.write("\n")
    print(f"\nDatos guardados en '{nombre_archivo}'")

# Guardar en archivo
guardar_en_archivo(diccionario_estudiantes, "estudiantes.txt")

def leer_archivo(nombre_archivo):
    diccionario = {}
    try:
        with open(nombre_archivo, "r", encoding="utf-8") as archivo:
            nombre = ""
            edad = 0
            materias = []
            for linea in archivo:
                linea = linea.strip()
                if linea.startswith("Nombre:"):
                    if nombre:  # guardar estudiante anterior antes de empezar uno nuevo
                        diccionario[nombre] = {
                            "edad": edad,
                            "materias": materias
                        }
                        materias = []
                    nombre = linea.replace("Nombre:", "").strip()
                elif linea.startswith("Edad:"):
                    edad = int(linea.replace("Edad:", "").strip())
                elif linea.startswith("-"):
                    materia = linea.replace("-", "").strip()
                    materias.append(materia)
            # Guardar el último estudiante
            if nombre:
                diccionario[nombre] = {
                    "edad": edad,
                    "materias": materias
                }
    except FileNotFoundError:
        print(f"No se encontró el archivo '{nombre_archivo}'.")
    return diccionario

    # Leer archivo y cargar los datos
archivo = "estudiantes.txt"
datos_cargados = leer_archivo(archivo)

# Mostrar los datos cargados
print("\n=== ESTUDIANTES CARGADOS DESDE ARCHIVO ===")
for nombre, datos in datos_cargados.items():
    print(f"\nNombre: {nombre}")
    print(f"Edad: {datos['edad']}")
    print("Materias:")
    for materia in datos["materias"]:
        print(f" - {materia}")
